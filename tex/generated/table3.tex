\begin{table}
    \footnotesize
    \centering
    \begin{threeparttable}
        \caption{Textual characteristics per sentence, by gender}
        \label{table3}
        \begin{tabular}{p{4cm}S@{}S@{}S@{}}
            \toprule
            &{Men}&{Women}&{Difference}\\
            \midrule
            \mrow{4cm}{No. characters}    &      134.72&      130.40&        4.32***\\
                                          &      (0.43)&      (1.45)&      (1.57)   \\
            \mrow{4cm}{No. words}         &       24.16&       23.09&        1.07***\\
                                          &      (0.08)&      (0.26)&      (0.29)   \\
            \mrow{4cm}{No. syllables}     &       40.65&       38.69&        1.96***\\
                                          &      (0.13)&      (0.45)&      (0.48)   \\
            \mrow{4cm}{No. polysyllabic words}&        4.69&        4.31&        0.38***\\
                                          &      (0.02)&      (0.07)&      (0.08)   \\
            \mrow{4cm}{No. difficult words}&        9.38&        8.91&        0.47***\\
                                          &      (0.03)&      (0.12)&      (0.13)   \\
            \bottomrule
        \end{tabular}
        \begin{tablenotes}
            \tiny
            \item \textit{Notes}. Sample 9,122 articles. Figures from an OLS regression of female ratio on each characteristic divided by sentence count. Male effects estimated at a ratio of zero; female effects estimated at a ratio of one. Robust standard errors in parentheses. ***, ** and * difference statistically significant at 1\%, 5\% and 10\%, respectively.
        \end{tablenotes}
    \end{threeparttable}
\end{table}
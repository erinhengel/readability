\begin{table}[H]
    \footnotesize
    \centering
    \begin{threeparttable}
        \caption{Journal readability, comparisons to \textit{AER}}
        \label{tableC3}
        \begin{tabular}{p{2cm}S@{}S@{}S@{}S@{}S@{}}
            \toprule
            &{\crcell[b]{Flesch\\[-0.1cm]Reading\\[-0.1cm]Ease}}&{\crcell[b]{Flesch-\\[-0.1cm] Kincaid}}&{\crcell[b]{Gunning\\[-0.1cm]Fog}}&{SMOG}&{\crcell[b]{Dale-\\[-0.1cm]Chall}}\\
            \midrule
            \textit{Econometrica}         &      -12.48***&       -4.44***&       -4.26***&       -2.63***&       -0.66***\\
                                          &      (1.93)   &      (0.41)   &      (0.47)   &      (0.38)   &      (0.16)   \\
            \textit{JPE}                  &       -5.69***&       -4.01***&       -3.42***&       -1.84***&        0.18   \\
                                          &      (1.93)   &      (0.41)   &      (0.47)   &      (0.38)   &      (0.16)   \\
            \textit{QJE}                  &        1.47** &       -0.04   &        0.28***&        0.19***&        0.27***\\
                                          &      (0.63)   &      (0.14)   &      (0.09)   &      (0.07)   &      (0.05)   \\
            \bottomrule
        \end{tabular}
        \begin{tablenotes}
            \tiny
            \item \textit{Notes}. Figures are the estimated coefficients on the journal dummy variables from (2) in~\autoref{table4}. Each contrasts the readability of the journals in the left-hand column with the readability of \textit{AER}. Standard errors clustered on editor in parentheses. ***, ** and * statistically significant at 1\%, 5\% and 10\%, respectively.
        \end{tablenotes}
    \end{threeparttable}
\end{table}
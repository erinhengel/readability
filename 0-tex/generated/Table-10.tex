\begin{table}
    \footnotesize
    \centering
    \begin{threeparttable}
        \caption{Readability of authors' \(t\)th paper (draft and final)}
        \label{table9}
        \begin{tabular}{p{3cm}S@{}S@{}S@{}S@{}S@{}S@{}}
            \toprule
            &{\(t=1\)}&{\(t=2\)}&{\(t=3\)}&{\(t=4\text{--}5\)}&{\(t\ge6\)}\\
            \midrule
            \multicolumn{6}{l}{\textbf{Predicted \(R_{jP}-R_{jW}\)}}\\
            \quad Women                   &        2.10***&        1.60** &        1.17   &        0.35   &       -0.18   \\
                                          &      (0.72)   &      (0.70)   &      (0.86)   &      (1.12)   &      (1.46)   \\
            \quad Men                     &       -0.21   &       -0.13   &        0.02   &       -0.21   &       -0.16   \\
                                          &      (0.18)   &      (0.10)   &      (0.09)   &      (0.14)   &      (0.17)   \\
            \midrule\multicolumn{6}{l}{\textbf{Marginal effect of female ratio}}\\
            \quad Published article       &        1.66   &        2.16** &        2.66***&        3.16***&        3.66** \\
                                          &      (1.11)   &      (0.85)   &      (0.86)   &      (1.14)   &      (1.54)   \\
            \quad Draft paper             &       -0.65   &        0.44   &        1.52** &        2.60***&        3.69***\\
                                          &      (1.30)   &      (0.93)   &      (0.75)   &      (0.88)   &      (1.23)   \\
            \midrule
            \textbf{Diff.-in-diff.}&        2.31***&        1.73** &        1.14   &        0.56   &       -0.02   \\
                                          &      (0.76)   &      (0.76)   &      (0.93)   &      (1.20)   &      (1.52)   \\
            \bottomrule
        \end{tabular}
        \begin{tablenotes}
            \tiny
            \item \textit{Notes}. Sample 4,289 observations. Panel one displays magnitude of predicted \(R_{jP}-R_{jW}\) (the direct effect of peer review) for women and men over increasing \(t\). Panel two estimates the marginal effect of an article's female ratio (\(\beta_1+\beta_2\times t\)), separately for draft papers and published articles. Figures from FGLS estimation of~\autoref{equation15}, weighted by \(N_{it}\) (see~\autoref{data}). Control variables include citation count (asinh), \(\text{max. }T\) (author prominence) and \(\text{max. }t\) (author seniority), native speaker and editor and journal-year fixed effects. "female ratio" defines papers with a strict minority of female authors as male-authored; for papers with 50 percent or more female authors, it is the ratio of female authors on a paper (see~\autoref{gender} for more details). Standard errors clustered by editor and robust to cross-model correlation in parentheses. ***, ** and * statistically significant at 1\%, 5\% and 10\%, respectively.
        \end{tablenotes}
    \end{threeparttable}
\end{table}
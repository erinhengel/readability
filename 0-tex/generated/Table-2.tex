\begin{table}
    \footnotesize
    \centering
    \begin{threeparttable}
        \caption{Textual characteristics per sentence by gender}
        \label{table2}
        \begin{tabular}{p{4cm}S@{}S@{}S@{}}
            \toprule
            &{Men}&{Women}&{Difference}\\
            \midrule
            \mrow{4cm}{No. characters}    &      134.72&      131.34&       -3.37** \\
                                          &      (0.42)&      (1.30)&      (1.37)   \\
            \mrow{4cm}{No. words}         &       24.15&       23.38&       -0.77***\\
                                          &      (0.08)&      (0.24)&      (0.25)   \\
            \mrow{4cm}{No. syllables}     &       40.64&       39.14&       -1.50***\\
                                          &      (0.13)&      (0.40)&      (0.42)   \\
            \mrow{4cm}{No. polysyllabic words}&        4.69&        4.40&       -0.29***\\
                                          &      (0.02)&      (0.06)&      (0.06)   \\
            \mrow{4cm}{No. difficult words}&        9.38&        9.03&       -0.35***\\
                                          &      (0.03)&      (0.10)&      (0.11)   \\
            \midrule
            No. observations              &       8,268&         849&       9,117   \\
            \bottomrule
        \end{tabular}
        \begin{tablenotes}
            \tiny
            \item \textit{Notes}. Figures are means of textual characteristics (per sentence) by sex. Male-authored papers are defined as having a ratio of female authors below 50 percent; female-authored papers are those with a ratio of female authors at or above 50 percent. Last column subtracts male means from female means. Standard errors in parentheses. ***, ** and * difference statistically significant at 1\%, 5\% and 10\%, respectively.
        \end{tablenotes}
    \end{threeparttable}
\end{table}
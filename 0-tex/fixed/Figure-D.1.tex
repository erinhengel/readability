\begin{figure}

	\floatbox{figure}[\FBwidth]
	{
		\caption{Readability score validity}\label{figure2}
	}
	{
	\begin{minipage}{0.95\linewidth}
	\begin{tabular}{cc}
		\multicolumn{2}{c}{\includegraphics[trim=0cm 2cm 0cm 1.9cm, clip,width=0.95\linewidth]{0-images/generated/{Figure-D.1-meta}.pdf}}\\
		\includegraphics[width=0.45\linewidth]{0-images/generated/{Figure-D.1-early}.pdf}&\includegraphics[width=0.45\linewidth]{0-images/generated/{Figure-D.1-late}.pdf}\\
	\end{tabular}
	\end{minipage}
		\floatfoot{\tiny\textit{Notes}. Top figure displays box plots of correlations between alternative measures of text difficulty and the Flesch Reading Ease, Flesch-Kincaid, Gunning Fog, SMOG and Dale-Chall readability scores. It includes 336 correlations found in 55 mostly peer reviewed papers. (See~\aref{appendixmetaanalysis} for the list of included studies and information on how they were selected.) Bottom figures plot abstracts' Flesch Reading Ease scores against their articles' citation counts (inverse hyperbolic sine (asinh) transformation) for the samples of top-four (excluding \textit{AER Papers \& Proceedings}) articles published before 1990 (left) and post-2000 (right). Each point represents the mean (in both dimensions) of roughly 170--180 observations. \(^\dagger\)Includes two studies which assessed readability using the Readability Assessment INstrument (RAIN), a comprehensive framework based on 14 variables, \textit{e.g.}, coeherence, writing style, illustrations and typography.}
	}
\end{figure}
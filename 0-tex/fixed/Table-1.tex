\begin{table}
	\footnotesize
	\centering
	\begin{threeparttable}
		\caption{Readability formulas}
		\label{tab:formulas}
		\begin{tabular}{ll}
			\toprule
			Score&Formula\\
			\midrule
			Flesch Reading Ease&$206.84-1.02\times\frac{\text{words}}{\text{sentences}}-84.60\times\frac{\text{syllables}}{\text{words}}$\\
			Flesch-Kincaid&$-15.59+0.39\times\frac{\text{words}}{\text{sentences}}+11.80\times\frac{\text{syllables}}{\text{words}}$\\
			Gunning Fog&$0.40\times\big(\frac{\text{words}}{\text{sentences}}+100\times\frac{\text{polysyllabic words}}{\text{words}}\big)$\\
			SMOG&$3.13+5.71\times\sqrt{\frac{\text{polysyllabic words}}{\text{sentences}}}$\\
			Dale-Chall&$3.64+0.05\times\frac{\text{words}}{\text{sentences}}+15.79\times\frac{\text{difficult words}}{\text{words}}$\\
			\bottomrule
		\end{tabular}
		\begin{tablenotes}
			\tiny
			\item \textit{Notes}. Table displays formulas used to calculate readability scores. Polysyllabic words refer to words with three or more syllables; difficult words are those not found on a list of 3,000 words understood by 80 percent of fourth-grade readers (aged 9--10)~\citep{Chall1995}.
		\end{tablenotes}
	\end{threeparttable}
\end{table}
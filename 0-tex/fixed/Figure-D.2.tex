\begin{figure}

	\floatbox{figure}[\FBwidth]
	{
		\caption{Abstract vs. article readability}\label{figure3}
	}
	{
	\begin{minipage}{0.95\linewidth}
	\begin{tabular}{cc}
		\includegraphics[width=0.45\linewidth]{0-images/generated/{Figure-D.2-fleschkincaid}.pdf}&\includegraphics[width=0.45\linewidth]{0-images/generated/{Figure-D.2-gunningfog}.pdf}\\
	\end{tabular}
	\end{minipage}
		\floatfoot{\tiny\textit{Notes}. Figures plot abstract readability against the readability of a 150--200 word passage of text from the introduction of the same paper. $\beta$ is the slope of the regression line (robust standard errors in parentheses). Sample only includes NBER Working Papers eventually published in a top-four economics journal with a heading explicitly titled ``Introduction'' (339 abstract-article pairs). Data are grouped into roughly 20 equal-sized bins; each point represents the mean (in both dimensions) of about 16--17 observations. Non-abstract text kindly provided by Henrik Kleven and Dana Scott~\citep{Kleven2018}. Readability scores calculated using the \texttt{R} \texttt{readability} package.}
	}
\end{figure}
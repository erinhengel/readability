\begin{table}[H]
	\footnotesize
	\centering
	\begin{threeparttable}
		\caption{Literature on gender differences in acceptance rates at peer reviewed journals}
		\label{AcceptanceRates}
		\begin{tabular}{L{3cm}L{3cm}L{8cm}}
			\toprule
			Study&Data source/context&Findings\\
			\midrule
			\citet{Card2020}&\textit{JEEA}, \textit{QJE}, \textit{Restat} and \textit{REStud}&Exclusively male- and female-authored manuscripts received a revise and resubmit decision 8 and 6 percent of the time, respectively.\\
			\citet{Blank1991}&\textit{AER}&12.7 and 10.6 percent of male- and female-authored papers were accepted at the \textit{AER}, respectively.\\
			\citet{Gilbert1994}&\textit{JAMA}&44.8 percent of referees accepted male-authored papers as is or if suitably revised; 29.6 percent summarily rejected them. Corresponding figures for female-authored papers were 38.3 and 33.3 percent, respectively.\\
			\citet{Handley2015}&American Fisheries Society journals&Female first-authored manuscripts were accepted 58.2 percent of the time; for male first-authored manuscripts, the corresponding figure was 62.5 percent. In no journal were female first-authored manuscripts accepted at higher rates than male first-authored manuscripts.\\
			\citet{McGillivray2018}&25 \textit{Nature}-branded journals&Among papers subjected to single-blind review, 23.7 percent of papers with a female corresponding author were sent out for review compared to 24.0 percent of papers with a male corresponding author. Conditional on being sent out for review, papers with a female corresponding author were accepted 44 percent of the time compared to 46 percent of papers with a male corresponding author.\\
			\citet{Nature2006}&\textit{Nature Neuroscience}&10.9 percent of papers with a female corresponding author were accepted compared to 11.8 percent of papers with a male corresponding author.\\
			\citet{Tregenza2002}&Four primary research journals&There was no statistically significant overall difference in acceptance rates by gender. The average per editor acceptance rate of papers with a male first author was 40.5 percent; the average per editor acceptance rate of papers with a female first author was 34.1 percent.\\
			% \citet{Budden2008}&\textit{Behavioral Ecology}&After double-blind review was introduced, there was a 7.9 percent increase in the proportion of papers with a female first author (\(p=0.01\)) and a similar decrease in papers with a male first author.\\
			\citet{Chari2017}&NBER Summer Institute&There was no gender difference in acceptance rates to NBER's Summer Institute.\\
			\citet{Ceci2014}&Comprehensive overview of the literature&``When it comes to actual manuscripts submitted to actual journals, the evidence for gender fairness is unequivocal: there are no sex differences in acceptance rates.'' (p. 111)\\
			\bottomrule
		\end{tabular}
		\begin{tablenotes}
			\item \textit{Notes}. Table summarises evidence on gender differences in acceptance rates at peer reviewed journals. (The list of included studies is likely not comprehensive.)
		\end{tablenotes}
	\end{threeparttable}
\end{table}
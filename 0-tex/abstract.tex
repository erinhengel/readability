% This paper documents several stylised facts about gender, readability and review times in top economics journals: (i) female authors are very under-represented; (ii) their papers are better written; (iii) the gender gap in readability widens during peer review; (iv) women improve their writing as they publish more papers but men do not; (v) female-authored papers take longer under review. Using a subjective expected utility framework, I argue that these stylised facts are consistent with editors and/or referees applying higher standards to women's writing. A counterfactual analysis suggests senior female economists may, as a result, write at least five percent more clearly than they otherwise would. As a final exercise, I show tentative evidence that women adapt to biased treatment in ways that may disguise it as voluntary choice.


Female authors are underrepresented in top economics journals. In this paper, I investigate whether higher writing standards contribute to the problem. I find: (i) female-authored papers are 1--6 percent better written than equivalent papers by men; (ii) the gap widens during peer review; (iii) women improve their writing as they publish more papers (but men do not); (iv) female-authored papers take longer under review. Using a subjective expected utility framework, I argue that higher writing standards for women are consistent with these stylised facts. A counterfactual analysis suggests senior female economists may, as a result, write at least five percent more clearly than they otherwise would. As a final exercise, I show tentative evidence that women adapt to biased treatment in ways that may disguise it as voluntary choice.